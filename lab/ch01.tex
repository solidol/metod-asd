\chapter{Основи OpenMP}
\nopagebreak[4]
\section*{Мета роботи}
Ознайомитись зi спецiфiкацiєю OpenMP. Навчитись розробляти та виконувати програми, що використовують OpenMP.

\nopagebreak[4]
\section{Теоретичнi вiдомостi}
\subsection*{Вступ}
\nopagebreak[4]

Одним зi стандартiв для програмування систем з загальною пам’яттю є iнтерфейс (API) OpenMP(Open Multi-Processing). Цей стандарт реалiзовано для мов програмування C, C++ та Fortran на великiй кiлькостi комп’ютерних архітектур, включаючи платформи Unix та Microsoft Windows.

OpenMP API складається з набору директив компiлятора pragma (прагм), функцiй та змiнних середовища, що впливають на поведiнку паралельної програми пiд час її виконання. Коли прагми OpenMP використовуються в програмi, вони дають вказiвки компiлятору створити виконуваний модуль, який буде виконуватись паралельно з викорстанням декiлькох потокiв.

В OpenMP застосовується модель паралельного виконання, що отримала назву «розгалудження-злиття». Така програма починається як один потiк виконання, який називають начальним потоком. Коли потiк зустрiчає паралельну конструкцiю, вiн створює нову групу потокiв, що складається з цього потока та позитивного числа допоміжних потокiв, i стає головним в новiй групi. Всi члени нової групи виконують код в межах паралельної конструкцiї. В кiнцi такої конструкцiї є неявний бар’єр. Пiсля її виконання код програми продовжує лише головний потiк.

\subsection*{Прагми OpenMP}
\nopagebreak[4]

Для активацiї пiдтримки прагм OpenMP в компiляторi необхiдно застосовувати додатковi параметри або флаги компiляцiї. Для Visual C++ таким параметром є /openmp, для gcc — \verb'-fopenmp'. Також у Visual C++ увiмкнути пiдтримку OpenMP можна за допомогою вiдповiдного параметру в дiалозi налаштування проекту: вибрати Configuration Properties, C/C++, Language i змiнити значення параметру «OpenMP Support» на yes. Для використання функцiй OpenMP необхiдно пiдключити до проекту файл \verb'vcomp.lib' (або \verb'vcompd.lib' в режимi вiдлагодження).

Прагми OpenMP починаються зi слiв \verb'#pragma omp' i мають наступний формат:

\verb'#pragma omp <директива> [список параметрiв]'

OpenMP пiдтримує наступнi директиви: atomic, barrier, critical, flush, for, master, ordered, parallel, parallel for, section, sections та single, що визначають або механiзми роздiлення коду, або конструцiї синхронiзацiї. Необов’язковi параметри директиви уточнюють її поведiнку.
Найбiльш уживана директива — parallel, яка має наступний синтаксис:
\verb'#pragma omp parallel [список параметрiв]'. Структурований блок
Вона iнформує компiлятор, що структурований блок (складений оператор) має виконуватись паралельно в кiлькох потоках. Як правило, кiлькiсть потокiв дорiвнює кiлькостi процесорiв в системi. В якостi прикладу розглянемо варiант класичної програми «Hello, world!» (див. \ref{code:hello}).

\begin{lstlisting}[label=code:hello,caption=Програма Hello OpenMP!]
#include <stdio.h>
int   main ( )
{
	#pragma omp   parallel
	{
		printf ( " Hello ,  OpenMP!\n" ) ;
	}   /* #pragma omp  parallel   */
	return   0; 
}   /*   int  main( )   */
\end{lstlisting}

На двопроцесорнiй системi результат роботи цiєї програми може бути наступним:

\begin{verbatim}
Hello, OpenMP! 
Hello, OpenMP!
\end{verbatim}

\subsection*{Функцiї та змiннi середовища OpenMP}
OpenMP пердбачає також набiр функцiй, що дозволяють:

\begin{itemize}
\item пiд час виконання програми отримувати та встановлювати рiзноманiтнi параметри, що визначають її поведiнку, наприклад, кiлькiсть потокiв, можливiсть вкладеного паралелiзму.
\item застосовувати синхронiзацiю на основi замкiв (locks).
Прототипи функцiй знаходяться в файлi omp.h. Розглянемо набiр функцiй OpenMP, що використовуються найчастiше.
\end{itemize}

Функцiя \verb'omp_get_thread_num()' повертає номер потоку, в якому була викликана. Головний потiк паралельного блоку має номер 0. Функцiя має наступний прототип:

\verb'int omp_get_thread_num(void)'

Функцiя \verb'omp_set_num_threads()' має наступний прототип:

\verb'void omp_set_num_threads(int num_threads)'

i використовується для встановлення кiлькостi потокiв, що будуть виконуватись в наступному паралельному блоцi. Для визначення поточної кiлькостi паралельних потокiв застосовується функцiя \verb'omp_get_num_threads()' з наступним прототипом:

\verb'int omp_get_num_threads(void)'
В лiстiнгу \ref{code:fun1} наведено текст програми, що демонструє використання розглянутих функцiй.
\begin{lstlisting}[label=code:fun1,caption=Функцiї OpenMP]
#include <stdio.h> 
#include <omp.h>
int   main ()
{
	omp_set_num_threads (5) ;
	#pragma omp   parallel
	{
		printf ( " Hello ,  OpenMP! ( thread  num=%d) \n", 
		omp_get_thread_num () ) ; 
	}   /* #pragma omp  parallel   */
	return   0;
}  /*   int  main()   */
\end{lstlisting}

В стандартi OpenMP визначено ряд змiнних середовища операцiйної системи, якi контролюють поведiнку OpenMP-програм. Зокрема, змiнна \verb'OMP_NUM_THREADS' визначає максимальну кiлькiсть потокiв, що будуть виконуватись в паралельнiй програмi. Наприклад, в командному рядку Windows необхідно виконати наступну команду:

\verb'set OMP_NUM_THREADS=4'

Таким чином, за допомогою функцiї \verb'omp_set_num_threads()' або змінної середовища \verb'OMP_NUM_THREADS' можна встановити довільну кiлькiсть потокiв, що будуть створюватись пiд час виконання OpenMP-програм.

\section{Індивідуальне завдання}
\nopagebreak[4]
\subsection*{Завдання до лабораторної роботи}
\nopagebreak[4]
\begin{enumerate}
\item Виконати програми, наведенi в лiстiнгах \ref{code:fun1} та \ref{code:hello}.
\item Змiнити програму з лiстiнгу \ref{code:fun1} таким чином, щоб кiлькiсть потокiв вказувалась користувачем програми.
\item Розробити програму, в якiй парнi потоки друкують «Hello, OpenMP!», а непарнi — iм’я та призвiще студента.

\end{enumerate}

\subsection*{Контрольні запитання}
\nopagebreak[4]
\begin{enumerate}
\item Для програмування яких обчислювальних систем — з розподiленною чи загальною пам’яттю — використовується стандарт OpenMP?
\item Назвiть складовi частини стандарту OpenMP.

\end{enumerate}



