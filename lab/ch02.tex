\chapter{Розпаралелювання коду в OpenMP}
\nopagebreak[4]
\section*{Мета роботи}
Ознайомитись з засобами OpenMP для розпаралелювання коду. Навчитись розробляти та виконувати програми, що використовують OpenMP.

\nopagebreak[4]
\section{Теоретичнi вiдомостi}
\subsection*{Директива for}
\nopagebreak[4]

Однiєю з директив, що застосовується досить часто, є директива for, яка належить до директив розподiлення роботи (work-sharing directive). Ця директива iнформує компiлятор, що при виконаннi циклу for в паралельному блоцi iтерацiї циклу мають бути розподiленi мiж потоками групи. Розглянемо роботу директиви for на прикладi програми, текст якої наведено в лiстiнгу \ref{code:for}.

\begin{lstlisting}[label=code:for,caption=Застосування директиви for]
#include <stdio.h>
#include <omp.h> 
int main() 
	{
	int i;
	#pragma omp parallel
		{ 
		#pragma omp  for
		for (i = 0;   i  <  5;   i++)
		printf ( "thread N%d i=%d\n", omp_get_thread_num () , i ) ;
		}  /* #pragma omp  parallel   */
	return 0;
}
\end{lstlisting}
Результат роботи програми наведено нижче:

\begin{verbatim}
thread N1 i=3 
thread N1 i=4 
thread N0 i=0 
thread N0 i=1 
thread N0 i=2
\end{verbatim}

Порядок появи рядкiв на екранi може бути довiльним i змiнюватись при кожному запуску програми. Якщо з тексту програми вилучити директиву \verb'#pragma omp for', то кожний потiк буде виконувати повний цикл for.

У зв’язку з тим, що розпаралелювання коду найчастiше виконується саме в циклiчних конструкцiях, OpenMP має скорочений варiант запису комбiнацiї директив  \verb'#pragma omp parallel' i \verb'#pragma omp for':
\verb'#pragma omp parallel for for(i = 0; i < 5; ++i)'


Слiд пiдкреслити, що розпаралелюватись можуть лише такi циклi, в яких iтерацiї не мають залежностей одна вiд одної. Якщо цикл не  має залежностей, компiлятор має змогу виконати цикл в любому порядку, навiть паралельно. Також цикли, якi OpenMP може розпаралелювати, мають вiдповiдати наступному формату:

\begin{itemize}
\item \verb'for(цiлий тип i = iнварiант циклу1;'
\item \verb'i {<,>,=,!=,<=,>=} iнварiант циклу2;' 
\item \verb'i {+,-}= iнварiант циклу3)'
\end{itemize}

Цi вимоги введенi для того, щоб OpenMP мiг визначити кiлькiсть iтерацiй циклу. При розробцi паралельних програм необхiдно враховувати, якi змiннi є спiльними (shared), а якi приватними (private). Спiльнi змiннi доступнi всiм потокам в групi, тому змiна значеннь таких змiнних в одному потоцi стає видимою в iнших потоках. Що стосується приватних змiних, то кожний потiк має окремi їх копiї, тому їх змiна в одному потоцi не вiдображається на iнших потоках. За замовчуванням всi змiннi в паралельному блоцi є спiльними. Лише в трьох випадках змiннi є приватними. По-перше, приватними є iндекси паралельних циклiв. По-друге, приватними є локальнi змiннi паралельних блокiв. По-третє, приватними стають змiннi, що вказанi в параметрах private, firstprivate, lastprivate и reduction прагми for.

Параметр private вказує, що для кожного потоку має бути створена локальна копiя кожної змiнної, що вказана в списку. Приватнi копiї будут iнiцiалiзуватись значенням за замовчанням, при можливостi буде застосовуватись конструктор за замовчанням.

Параметр firstprivate виконує аналогiчнi дiї, однак перед виконанням паралельного блоку вiн копiює значення приватної змiнної в кожний потiк, застосовуючи при необхiдностi конструктор копiй.

Параметр lastprivate теж подiбний до параметра private, однак пiсля виконання останньої iтерацiї циклу або конструкцiї паралельного блоку значення змiнних, що вказанi в списку цього параметра, привласнюються змiнним основного потоку. При цьому може застосовуватись оператор привласнювання копiй.
Параметр reduction в якостi аргументiв приймає змiнну й оператор. Оператори, що пiдтримуються reduction, наведенi в табл. \ref{tab:table1}, а змiнна має бути скалярною (наприклад, int або float). Змiнна параметру reduction iнiцiалiзується значенням, що вказане в табл. \ref{tab:table1}. Пiсля завершення паралельного блоку вказаний оператор застосовується до кожної приватної копiї i початкового значеннi змiнної.

	
\begin{longtable}[t]{|l|p{20em}|}

\caption{Оператори параметру reduction} \label{tab:table1}\\
\hline

Оператор reduction & Iнiцiалiзацiйне значення змiнної \\
\hline


\verb|+| & 0 \\
\verb|*| & 1 \\
\verb|-| & 0 \\
\verb|&| & \~0 (усi бiти в змiннiй встановленi) \\
\verb'|' & 0 \\
\verb|^| & 0 \\
\verb|&&| & 0 \\
\verb|||| & 0 \\

\hline
\end{longtable}

В лiстiнгу \ref{code:for1} наведено програму, що використовує розглянутi параметри. 

\begin{lstlisting}[label=code:for1,caption=Параметри прагми for]
#include <stdio . h>
#include <omp.h> 
int   main () 
{
int i, j;
int array[] = {1, 2, 3, 4, 5, 6, 7, 8, 9};
int sum; 
#pragma omp   parallel
{ 
	#pragma omp for firstprivate(j)   \
	lastprivate(i)   \
	reduction (+: sum)
	for (i = 0; i<sizeof (array)/ sizeof (array[0]); i++)
	{
		long   private = i; 
		sum += array[i];
	}   /*	for(i = 0;   i  <  5;   i++) */
	}  /* #pragma omp  parallel   */
printf("%d\n", sum);
return 0; 
}
\end{lstlisting}
В цьому лiстiнгу змiнна i є приватною тому, що вона є змiнною циклу for, а також тому, що вказана в параметрi lastprivate. Змiнна j примусово зроблена приватною за допомогою firstprivate. Змiнна private також є приватною через те, що вона оголошена в паралельному блоцi. В кожному потоцi екземпляр приватної змiнної sum неявно iнiцiалiзується нулем.

\subsection*{Директива sections}
Як правило, OpenMP застосовується для розпаралелювання циклiв. Однак, в OpenMP є також засоби для пiдтримки паралелiзму на рiвнi функцiй. Цей механiзм називається секцiями OpenMP (OpenMP sections). Для створення паралельного блоку секцiй застосовується директива \verb'#pragma omp sections', або, аналогiчно директивi \verb'#pragma omp parallel for', її скорочений варiант \verb'#pragma omp parallel sections'. Секцiї в блоцi створюються директивой \verb'#pragma omp section'. Кожнiй секцiї ставиться у вiдповiднiсть один потiк, i всi секцiї виконуються паралельно. Фрагмент програми, що демонструє розпаралелювання на рiвнi функцiй наведено в лiстiнгу \ref{code:sect1}. 

\begin{lstlisting}[label=code:sect1,caption=Директива omp section]
void quickSort(int L, int R) 
{
int i;
if (R > L)
	{
	i = partition (L,  R); 
	#pragma omp parallel sections
		{
		#pragma omp section
		quickSort(L,   i -  1);
		#pragma omp  section
		quickSort( i  + 1, R);
		}   /* #pragma omp   parallel   sections   */
	}   /*	if   (R>  l)   */
}   /*   void   quickSort(int   l ,   int   r)   */
\end{lstlisting}

\section{Індивідуальне завдання}
\nopagebreak[4]
\subsection*{Завдання до лабораторної роботи}
\nopagebreak[4]
\begin{enumerate}
\item Виконати програми, наведенi в лiстiнгах \ref{code:for1} та \ref{code:sect1}.
\item Розробити програму для обчислення добутку елементiв масиву.
\item Написати програму, що складається з двох потокiв, з яких один обчислює суму, а iнший добуток елементiв масиву. Для створення потокiв використовувати механiзм секцiй.


\end{enumerate}

\subsection*{Контрольні запитання}
\nopagebreak[4]
\begin{enumerate}
\item Якими директивами виконується розпаралелювання циклiв в OpenMP?
\item Якi змiннi в паралельних блоках є спiльними, а якi — приватними?
\item Яким чином дiє параметр \verb'private'?
\item Яким чином дiє параметр \verb'firstprivate'?
\item Яким чином дiє параметр \verb'lastprivate'?
\item Яким чином дiє параметр \verb'reduction'?
\item Якi оператори можуть застосовуватись з параметром \verb'reduction'?
\item Яким чином виконується розпаралелювання коду на рiвнi \verb'функцiй'?
\end{enumerate}



