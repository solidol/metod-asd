\chapter{Рекурсія і рекурсивні алгоритми}
\nopagebreak[4]
\section*{Мета роботи}
Вивчити поняття, види рекурсії і рекурсивну тріаду, навчитися розробляти рекурсивну тріаду при вирішенні завдань мовою C++.

При виконанні лабораторної роботи для кожного завдання потрібно написати програму на мові С++, яка отримує на вході числові дані, виконує їх обробку відповідно до вимог завдання і виводить результат на екран. Для обробки даних необхідно реалізувати рекурсивну функцію. Введення даних здійснюється з клавіатури з урахуванням вимог до вхідних даних, що містяться в постановці завдання (введення даних супроводжуйте діалогом). Обмеженнями на вхідні дані є допустимий діапазон значень використовуваних числових типів в мові С++.

\nopagebreak[4]
\section{Теоретичнi вiдомостi}

\begin{lstlisting}[label=code:rec1,caption=Програма обчислення факторіалу ітераційним способом]
public class App {
    public static void main(String[] args) {
        System.out.println(factorial(5));
    }
     
    public static int factorial(int arg) {
        int result = 1;
        for (int k = 1; k <= arg; k++) {
            result *= k;
        }
        return result;
    }
}
 
>> 120
\end{lstlisting}

\begin{lstlisting}[label=code:rec1,caption=Програма обчислення факторіалу рекурсивним способом]
public class App {
    public static void main(String[] args) {
        System.out.println(factorial(5));
    }
     
    public static int factorial(int arg) {
        if (arg == 1) {
            return 1;
        } else {
            return arg * factorial(arg - 1);
        }
    }
}
 
>> 120
\end{lstlisting}
\nopagebreak[4]
\section{Вказівки до виконання роботи.}
\nopagebreak[4]

Кожне завдання необхідно вирішити вивченими рекурсивними методами вирішення завдань і методами обробки числових даних у мові С++. Перед реалізацією коду кожного завдання необхідно розробити рекурсивну тріаду відповідно до постановкою завдання: виконати параметризацію, виділити базу і оформити декомпозицію рекурсії. Програму для вирішення кожного завдання необхідно розробити методом процедурної абстракції, використовуючи рекурсивні функції. Етапи супроводити коментарями в коді.

Слід реалізувати кожне завдання у відповідності з наведеними етапами:

\begin{itemize}
\item вивчити словесну постановку задачі, виділивши при цьому всі види даних;
\item сформулювати математичну постановку задачі;
\item вибрати метод розв'язання задачі, якщо це необхідно;
\item розробити графічну схему алгоритму;
\item записати розроблений алгоритм на мові С ++;
\item розробити контрольний тест до програми;
\item налагодити програму;
\item подати звіт по роботі. 
\end{itemize}



\section{Індивідуальне завдання}
\nopagebreak[4]
\subsection*{Завдання до лабораторної роботи}
\nopagebreak[4]
\begin{enumerate}
\item Створіть програму для обчислення $n$-го числі Фібоначчі.

\end{enumerate}

\subsection*{Контрольні запитання}
\nopagebreak[4]
\begin{enumerate}
\item Чи можна випадок непрямої рекурсії звести до прямої рекурсії? Відповідь обґрунтуйте.
\item Чи може рекурсивна база містити кілька тривіальних випадків? Відповідь обґрунтуйте.
\item Чи є параметри, база і декомпозиція єдиними для конкретного завдання? Відповідь обґрунтуйте.
\item З якою метою в задачах відбувається перегляд або коригування обраних параметрів, виділеної бази або випадку декомпозиції?
\item Чи є рекурсія універсальним способом вирішення завдань? Відповідь обґрунтуйте.
\item Чому для оцінки трудомісткості рекурсивного алгоритму недостатньо одного методу підрахунку вершин рекурсивного дерева?


\end{enumerate}



