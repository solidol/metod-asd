\chapter{Правила оформлення звіту}
\section{Титульна сторінка лабораторної роботи}
\input{ap01-00}
\section{Приклади блок-схем}
Правила виконання блок-схем задані наступними документами:
\begin{itemize}
\item ГОСТ 19.701-90. Схемы алгоритмов, программ, данных и систем. Условные обозначения и правила выполнения
\item ГОСТ 19.002-80. Схемы алгоритмов и программ. Правила выполнения
\item ГОСТ 19.003-80. Схемы алгоритмов и программ. Обозначения условные графические

\end{itemize}


\chapter{Лістинги}
\section{Програмна реалізація алгоритму Хаффмана за допомогою кодового дерева}

\begin{lstlisting}[label=code:haff,caption=Програмна реалізація алгоритму Хаффмана]
#include "stdafx.h"
#include <iostream>
using namespace std;
struct sym { 
      unsigned char ch;
      float freq;     
      char code[255];
      sym *left;
      sym *right;
};
void Statistics(char *String);
sym *makeTree(sym *psym[],int k);
void makeCodes(sym *root);
void CodeHuffman(char *String,char *BinaryCode, sym *root);
void DecodeHuffman(char *BinaryCode,char *ReducedString, 
                   sym *root);
int chh;//переменная для подсчета информация из строки
int k=0;
//счётчик количества различных букв, уникальных символов
int kk=0;//счетчик количества всех знаков в файле
int kolvo[256]={0};
//инициализируем массив количества уникальных символов
sym simbols[256]={0};//инициализируем массив записей 
sym *psym[256];//инициализируем массив указателей на записи
float summir=0;//сумма частот встречаемости

int _tmain(int argc, _TCHAR* argv[]){
  char *String = new char[1000];
  char *BinaryCode = new char[1000];
  char *ReducedString = new char[1000];
  String[0] = BinaryCode[0] = ReducedString[0] = 0;
  cout << "Введите строку для кодирования : ";
  cin >> String;
  sym *symbols = new sym[k];
  //создание динамического массива структур simbols
  sym **psum = new sym*[k];
  //создание динамического массива указателей на simbols
  Statistics(String);
  sym *root = makeTree(psym,k);
  //вызов функции создания дерева Хаффмана    
  makeCodes(root);//вызов функции получения кода
  CodeHuffman(String,BinaryCode,root);
  cout << "Закодированная строка : " << endl;
  cout << BinaryCode << endl;
  DecodeHuffman(BinaryCode,ReducedString, root);
  cout << "Раскодированная строка : " << endl;
  cout << ReducedString << endl;
  delete psum;
  delete String;
  delete BinaryCode;
  delete ReducedString;
  system("pause");
  return 0;
}
//рeкурсивная функция создания дерева Хаффмана
sym *makeTree(sym *psym[],int k) { 
  int i, j;
    sym *temp;
    temp = new sym;
    temp->freq = psym[k-1]->freq+psym[k-2]->freq;
    temp->code[0] = 0;
    temp->left = psym[k-1];
    temp->right = psym[k-2];
    if ( k == 2 )
        return temp;
    else {
    //внесение в нужное место массива элемента дерева Хаффмана
      for ( i = 0; i < k; i++)
        if ( temp->freq > psym[i]->freq ) {
          for( j = k - 1; j > i; j--)
            psym[j] = psym[j-1]; 
            psym[i] = temp;
            break;
        } 
    }
  return makeTree(psym,k-1);
}
//рекурсивная функция кодирования дерева
void makeCodes(sym *root) { 
  if ( root->left ) {
    strcpy(root->left->code,root->code);
    strcat(root->left->code,"0");
    makeCodes(root->left);
  }
  if ( root->right ) {
    strcpy(root->right->code,root->code);
    strcat(root->right->code,"1");
    makeCodes(root->right);
  }
}
/*функция подсчета количества каждого символа и его вероятности*/
void Statistics(char *String){
  int i, j;
  //побайтно считываем строку и составляем таблицу встречаемости 
  for ( i = 0; i < strlen(String); i++){
    chh = String[i];
    for ( j = 0; j < 256; j++){
      if (chh==simbols[j].ch) {
        kolvo[j]++;
        kk++; 
        break;
      }
      if (simbols[j].ch==0){
        simbols[j].ch=(unsigned char)chh;
        kolvo[j]=1;
        k++; kk++;
        break;
      } 
    } 
  }
  // расчет частоты встречаемости
  for ( i = 0; i < k; i++)
    simbols[i].freq = (float)kolvo[i] / kk;
  // в массив указателей заносим адреса записей
  for ( i = 0; i < k; i++) 
    psym[i] = &simbols[i];
  //сортировка по убыванию 
  sym tempp;
  for ( i = 1; i < k; i++)
  for ( j = 0; j < k - 1; j++)
    if ( simbols[j].freq < simbols[j+1].freq ){
      tempp = simbols[j];
      simbols[j] = simbols[j+1];
      simbols[j+1] = tempp;
    }
  for( i=0;i<k;i++)  {
    summir+=simbols[i].freq; 
    printf("Ch= %d\tFreq= %f\tPPP= %c\t\n",simbols[i].ch,
            simbols[i].freq,psym[i]->ch,i);
  }
  printf("\n Slova = %d\tSummir=%f\n",kk,summir);
}
//функция кодирования строки
void CodeHuffman(char *String,char *BinaryCode, sym *root){
  for (int  i = 0; i < strlen(String); i++){
    chh = String[i];
    for (int  j = 0; j < k; j++)
      if ( chh == simbols[j].ch ){
        strcat(BinaryCode,simbols[j].code);
      }
  }
}
//функция декодирования строки
void DecodeHuffman(char *BinaryCode,char *ReducedString, 
                   sym *root){
  sym *Current;// указатель в дереве
  char CurrentBit;// значение текущего бита кода
  int BitNumber;
  int CurrentSimbol;// индекс распаковываемого символа
  bool FlagOfEnd;  // флаг конца битовой последовательности
  FlagOfEnd = false;
  CurrentSimbol = 0;
  BitNumber = 0;
  Current = root;
  //пока не закончилась битовая последовательность
  while ( BitNumber != strlen(BinaryCode) ) {
    //пока не пришли в лист дерева
    while (Current->left != NULL && Current->right != NULL && 
           BitNumber != strlen(BinaryCode) ) {
      //читаем значение очередного бита
      CurrentBit = BinaryCode[BitNumber++];
      //бит – 0, то идем налево, бит – 1, то направо
      if ( CurrentBit == '0' ) 
        Current = Current->left;
      else 
        Current = Current->right;
    }
    //пришли в лист и формируем очередной символ
    ReducedString[CurrentSimbol++] = Current->ch;
    Current = root;
  }
  ReducedString[CurrentSimbol] = 0;
}
\end{lstlisting}


\section{Опис функції сортування простим злиттям}

\begin{lstlisting}[label=code:sort111,caption=Програмна реалізація функції сортування простим злиттям]
//Описание функции сортировки простым слиянием
void Simple_Merging_Sort (char *name){
  int a1, a2, k, i, j, kol, tmp;
  FILE *f, *f1, *f2;
  kol = 0;
  if ( (f = fopen(name,"r")) == NULL )
    printf("\nИсходный файл не может быть прочитан...");
  else {
    while ( !feof(f) ) {
      fscanf(f,"%d",&a1);
      kol++;
    }
    fclose(f);
  }
  k = 1;
  while ( k < kol ){
    f = fopen(name,"r");
    f1 = fopen("smsort_1","w");
    f2 = fopen("smsort_2","w");
    if ( !feof(f) ) fscanf(f,"%d",&a1);
    while ( !feof(f) ){
      for ( i = 0; i < k && !feof(f) ; i++ ){
        fprintf(f1,"%d ",a1);
        fscanf(f,"%d",&a1);
      }
      for ( j = 0; j < k && !feof(f) ; j++ ){
        fprintf(f2,"%d ",a1);
        fscanf(f,"%d",&a1);
      }
    }
    fclose(f2);
    fclose(f1);
    fclose(f);

    f = fopen(name,"w");
    f1 = fopen("smsort_1","r");
    f2 = fopen("smsort_2","r");
    if ( !feof(f1) ) fscanf(f1,"%d",&a1);
    if ( !feof(f2) ) fscanf(f2,"%d",&a2);
    while ( !feof(f1) && !feof(f2) ){
      i = 0;
      j = 0;
      while ( i < k && j < k && !feof(f1) && !feof(f2) ) {
        if ( a1 < a2 ) {
          fprintf(f,"%d ",a1);
          fscanf(f1,"%d",&a1);
          i++;
        }
        else {
          fprintf(f,"%d ",a2);
          fscanf(f2,"%d",&a2);
          j++;
        }
      }
      while ( i < k && !feof(f1) ) {
        fprintf(f,"%d ",a1);
        fscanf(f1,"%d",&a1);
        i++;
      }
      while ( j < k && !feof(f2) ) {
        fprintf(f,"%d ",a2);
        fscanf(f2,"%d",&a2);
        j++;
      }
    }
    while ( !feof(f1) ) {
      fprintf(f,"%d ",a1);
      fscanf(f1,"%d",&a1);
    }
    while ( !feof(f2) ) {
      fprintf(f,"%d ",a2);
      fscanf(f2,"%d",&a2);
    }
    fclose(f2);
    fclose(f1);
    fclose(f);
    k *= 2;
  }
  remove("smsort_1");
  remove("smsort_2");
}
\end{lstlisting}