

1. Систення інформація за допомогою бінарних дерев.
2. Представлення виразів за допомогою дерев. СР1 Алгоритм Хафмена
3. Представлення багатовітвлевих дерев.
4. Представлення графів. СР2 Граф у вигляді списку
5. Алгоритми на графах. СР3 Матриця шляхів СР4 Пошук циклів у графі
6. Використання багатовітвлевих дерев. 
7. Алгоритми сортування СР4 Загальні види сортувань
8. Алгоритми внутрішнього сортування СР5 Сортування Шелла та  деревовидне сортування
9. Алгоритми зовнішнього сортування СР6 Сортування злиттям
10. Характеристика алгоритмів порівняння методів сортування СР7 О-складність алгоритма
11. Алгоритми розподілу обчислювального процессу
12. Архітектура розподіленних обчислень СР 8 Архітектура Фон Неймана
13. Процеси і потоки в обчисленні
14. Реалізація багатозадачного середовища СР8 Моделі багатозадачності
15. Бібліотеки організації розподілених обчислень СР 9 Бібліотеки Опен МП та МПІ
16. Програмна модель ОпенМП СР10 Основні синтаксичні конструкції ОпенМП 
17. Конструкції ОпенМП  для розподілу робіт
18. Умови виконання (планування) СР11 Паралельні секції
19. Бібліотечні функції ОпенМП  СР12 Змінні середовища ОпенМП 
20. Бібліотека МПІ
21. Режими обліку повідомлень в МПІ


---
Это сообщение проверено на вирусы антивирусом Avast.
https://www.avast.com/antivirus

