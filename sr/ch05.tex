\chapter{Основи мережі Internet}
\nopagebreak[4]
\section*{Мета роботи}
Вивчити основи роботи мережі Інтернет. Ознайомитись з роботою протоколу НТТР. Навчитися встановлювати необхідне програмне забезпечення для розробки сценаріїв на РНР.
\nopagebreak[4]
\section{Структура мережі Internet}
\nopagebreak[4]


\section{Індивідуальне завдання}
\nopagebreak[4]
\subsection*{Завдання до лабораторної роботи}
\nopagebreak[4]
\begin{enumerate}
\item Вивчити теоретичний матеріал
\item Відповісти на контрольні запитання
\item Скласти звіт
\item Захистити роботу
\end{enumerate}

\subsection*{Контрольні запитання}
\nopagebreak[4]
\begin{enumerate}
\item Що таке Internet? З яких структурних частин складається Internet?
\item Що таке IP-адреса?
\item Що таке доменне ім'я, з чого воно складається?
\item Який сервіс Internet перетворює IP-адреси в доменні імена і навпаки?
\item Яка служба займається розподіленням блоків IP-адрес?
\item Протокол HTTP. Рівень у моделі OSI, призначення.
\item Значення URI, URL, URN.
\item Мови web-програмування, які ви знаєте.
\item Веб-сервери, які ви знаєте.
\item Мережеві СКБД, які ви знаєте. 
\end{enumerate}



